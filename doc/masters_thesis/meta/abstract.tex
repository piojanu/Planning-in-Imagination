\section*{Abstract}

The aim of this thesis is to derive from previous work on model learning in complex high-dimensional decision making problems and apply them to planning in complex tasks. Those methods proved to train accurate models, at least in short horizon, and should open a path for application of planning algorithms to i.e. Atari 2600 games, a platform used for evaluation of general competency in artificial intelligence. The goal is to improve data efficiency without loss in performance compared to model-free methods. This work focuses on three benchmarks: an arcade game with dense rewards Boxing, a challenging environment with sparse rewards Freeway and a complex puzzle game Sokoban.

Chapter 1 outlines the context and purpose of this work, underlying the significance of general planning artificial intelligence system. 

Chapter 2 describes the problem specification in greater detail and presents the required theoretical background. It touches topics like: decision making formalism, reinforcement learning, simulation-based search, deep learning, model learning and model architectures.

Chapter 3 presents current state of knowledge, related to this thesis problem, in the reinforcement learning field. It also includes high-level overview of methods that serve as a foundation of this thesis solutions.

Chapter 4 describes technical details. It considers the framework, hardware and benchmarks used during research.

Chapter 5 describes this thesis solutions, presenting their: architecture, planning algorithms, preprocessing and data collection methods and implementation details in depth.

Chapter 6 presents conducted experiments and discuss their results. It demonstrates the progress of developed solutions as well as the process of searching for optimal hyper-parameters.

Chapter 7 summarizes this work, drawing conclusions from the research and experiments performed. It also suggest actions for future development.

\vspace{1cm}
\noindent
\textbf{Keywords:} computer science, deep learning, reinforcement learning, planning and learning, search-based simulation, Arcade Learning Environment

\vspace{0.5cm}
\noindent
\textbf{OECD field of science and technology:} Computer and information sciences
