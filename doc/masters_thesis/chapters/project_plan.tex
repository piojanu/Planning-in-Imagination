\section{Planning with learned model}

In this section two architectures are described. Both involve a similar model learning approach, but differ substantially in technical details and planning algorithms. Nevertheless, the goal stays the same: train a sufficient environment model, or such that accurately predicts future latent states of an environment to predefined cut-off point in time, and use it to plan and solve the environment. The more accurate the model to the cut-off point, which might be environment dependent, the better model learning algorithm.
Before all of that, the code architecture and the framework that was created to accelerate this research are described.

\subsection{HumbleRL framework}

Reinforcement Learning scientists tend to write the entire code from scratch by themselves, instead of using existing RL frameworks. This is justified by the fact, that the commonly available frameworks are not flexible enough for intended experiments or require a specific backend like TensorFlow, which might be disfavored.
HumbleRL \cite{Code.HRL} was created with this problem in mind. Its simple API allows to perform a variety of RL experiments without any restrictions on the algorithms used. Since the backend is not tied to any specific technology, it is possible to mix different neural network frameworks or not use any at all. HumbleRL provides the boilerplate code of RL loop in fig.\ref{Fig.RL} and determines the common interface, the rest is done by the user.

\subsubsection{Architecture}
Framework architecture is depicted in fig. \ref{Fig.HRL_architecture}. An agent is represented by the Mind class. Mind encapsulates action planning logic and provides it via the plan method. In order to learn, the agent acts in the world represented by the Environment class. The Environment class provides methods for resetting, taking steps, rendering and getting information about the world. The agent is not usually presented with raw environment observations. Instead, it looks at states preprocessed by the Interpreter. Different interpreters can be joined together with the ChainInterpreter class. It acts as a preprocessing pipeline, with each subsequent interpreter using the output of a previous one as an input.

\begin{figure}[H]
\includegraphics[width=0.45\textwidth,height=0.9\textheight,keepaspectratio]{figures/HumbleRL/architecture.png}
\caption{HumbleRL architecture}
\label{Fig.HRL_architecture}
\end{figure}

Framework user does not need to call all of those methods directly, those are utilized by the loop function. This function gets an action from the Mind, executes it in the Environment and then next observation is preprocessed with the Interpreter in preparation for the next step. To extend basic loop functionality, user can define callbacks that implement the Callback interface. Callbacks can react to events:
\begin{itemize}
\item at the beginning and ending of the loop,
\item at the beginning and ending of each episode,
\item after action is planned by the Mind,
\item after step is taken in the Environment.
\end{itemize}
Callbacks are accumulated in the CallbackList. The entire loop function logic is shown in fig. \ref{Fig.HRL_loop}.
Parallel version of loop function is available as the pool function. It uses predefined number of workers to execute a pool of Minds in their own Environments in parallel.

\begin{figure}[H]
\includegraphics[width=0.45\textwidth,height=0.9\textheight,keepaspectratio]{figures/HumbleRL/loop.png}
\caption{HumbleRL loop function overview}
\label{Fig.HRL_loop}
\end{figure}

World Models with the AlphaZero planner uses this framework.

\subsection{World Models with the AlphaZero planner}

\begin{enumerate}
\item High-level idea and argument ``why?!'':\\
  World Models shown it can plan (offline planning, training a policy) in imagination and AlphaZero is incredibly powerful search algorithm.
\item Data collection i.e. a random agent.
\item Preprocessing.
\item World Models architecture: Vision and Memory.
  \begin{enumerate}
  \item Which does what (brief reminder) e.g. Vision encode a current observation and Memory encodes history and predicts future.
  \item How Memory is used to predict future: MDN and LMH.
  \item Training procedure of each module i.e. in separation exactly like described in the related work (don't describe loss etc. only high-level).
  \end{enumerate}
\item AlphaZero architecture: Controller.
  \begin{enumerate}
  \item What it does i.e. uses the memory module to plan in imagination.
  \item How next action is planned i.e. MCTS.
  \item Training procedure of Value and Policy networks i.e. policy iteration after Vision and Memory are already trained (don't go into AZ details, it's already in the related work, here how you modify it or apply to your case).
  \end{enumerate}
\end{enumerate}

World Models' agent, as shown in the paper \cite{Algo.WorldModels}, is able to learn from simulated experience. It is an example of successful planning using learned model. This section describes attempt to utilize the world model part of the agent in the AlphaZero search algorithm. Moreover, the Vision module encodes environment observations into low level representation. The latent state of the world models encodes abstract information about the environment and allow the planning controller to work fast, because there is no need to generate high-dimensional observations, only low-dimensional latent states are generated by the model and processed by the controller.

\subsubsection{Data collection}
To train Vision and Memory modules first collection of 10,000 random rollouts of the environment are gathered to create a dataset. An agent is acting randomly to explore the environment multiple times and records of the random actions taken and the resulting observations from the environment are saved.
This dataset is used to train the Vision module to learn a latent state of each frame observed. Next, it is used to preprocess each frame into its latent state to prepare a dataset for the Memory module training.

\subsubsection{Preprocessing}
Each frame, before it is used for any training, is central cropped if a frame from an environment includes e.g. some border. This operation depends on a specific environment. It is then resized to 64 x 64 pixels for all environments. All three colour channels are preserved. Actions get one-hot encoded.

\subsubsection{World Models architecture: Vision and Memory}
HumbleRL was used in a few stages of training a model. First of all, OpenAI Gym was used behind the Environment interface from the framework. Loop together with an agent performing random actions (Mind) and a callback was used to gathers transitions and save them to an external storage. The framework allows to focus strictly on collecting trajectories and not worry about agent-environment interactions.
Transitions are used to train the Vision and Memory components. For Vision, which function is to encode observations into latent state-space, Keras \cite{Code.Keras} framework was used and for Memory, which function is to encode history, PyTorch \cite{Code.PyTorch} was used, since it was easier to use for this case than Keras. HumbleRL is not constricted to work with any particular deep learning library, so it’s not a problem to mix the solutions, as long as trained models are wrapped in proper interfaces.
As described in the related work chapter, the Vision module was trained to encode each frame into low dimensional latent vector by minimizing the difference between a given frame and the reconstructed version of the frame produced by the decoder. After data processing using the trained Vision model, the Memory module can be trained to model environment's dynamics, predicting a future latent state from history of previous latent states, as a mixture of Gaussians.
Figure \ref{Fig.WorldModelsPHM} depicts how the world model works. Solid arrows describe the predictive model and dotted arrow describes the inference model \editnote{NOTE: It needs to be explained in Theoretical Background what is the difference.}. This model uses stochastic nodes (circles) to model uncertainty in the environment, but also create more robust environment's representation. Uncertainty can originate not only from fundamental stochastic nature of the environment, but also partial observability. Squares depict deterministic nodes. The Memory module model is implemented as the recurrent neural network.

\begin{figure}[H]
\includegraphics[width=0.45\textwidth,keepaspectratio]{figures/WorldModels/prob_graph_model.png}
\caption{Stochastic World Models probabilistic graphical model diagram}
\label{Fig.WorldModelsPHM}
\end{figure}
\editnote{TODO: Describe this diagram in probabilistic notation (see note).}

Because benchmarks include deterministic and fully observable case, World Models was tested also with the Memory module without Mixture Density Network on top of the recurrent neural network. Instead a linear model was used to output the next latent state.

\subsubsection{AlphaZero architecture: Controller}
Once the world model is ready it is used to train the final piece - the planner, which uses the AlphaZero algorithm. The Vision module is used as the Interpreter which encodes incoming observations into latent space. The Memory module gets wrapped in the MDP interface, it is used for AlphaZero simulations. The Mind interface is implemented by AlphaZero algorithm, it implements described in the related work search strategy. It uses a neural network to evaluate each node and guide search direction on each edge. Returned actions' scores are visit counts from the root node, which are then used to choose an action by a policy. Details are the same as in original AlphaZero described in the related work. Pseudo-code of the planning algorithm in Python is shown below.

\begin{lstlisting}[language=Python]
def plan(self, state):
    # Get/create root node
    root = self.query_tree(state)

    # Perform simulations
    simulations = 0
    start_time = time()
    while time() < start_time + self.timeout and simulations < self.max_simulations:
        # Simulate
        simulations += 1
        leaf, path = self.simulate(root)

        # Expand and evaluate
        value = self.evaluate(leaf)

        # Backup value
        self.backup(path, value)

    # Get actions' visit counts
    actions = np.zeros(self.model.action_space.num)
    for action, edge in root.edges.items():
        actions[action] = edge.num_visits

    return actions
\end{lstlisting}

Agent's experience and score statistics used for training are gathered using callbacks during self-play phase. The neural network training phase takes place after a predefined number of self-play games. The training phase is performed using the Keras \cite{Code.Keras} framework. Next, the self-play phase takes place once again and the two further interchange. The self-play phase uses the loop function to effortlessly run AlphaZero for given number of games (episodes).

\subsection{PlaNet with the CEM planner}

\begin{enumerate}
\item High-level idea and argument ``why?!'':\\
  PlaNet shown it can successfully plan (online planning, evolutional strategy) in imagination for complex continuous control task with iterative data collection and in a clean algorithm.
\item Data collection i.e. iterative approach.
\item Preprocessing.
\item RSSM architecture.\\ 
  \begin{enumerate}
  \item What it does (brief reminder)? It predicts future, observations and rewards, where the latter is more important for planner which uses the model to evaluate plans. 
  \item How it's used. Paper details are already described in the related work (like overshooting etc.), here write how you use it e.g. you encode actions and RSSM is used to sample future latent states that are then used to predict rewards and it receives current observation to update its belief state etc.
  \item Training procedure i.e. interchanged training, test and collection phases (like in paper, nothing changed).
  \end{enumerate}
\item CEM planner.
  \begin{enumerate}
  \item What it does i.e. it's like optimization of actions scores based on model evaluations.
  \item How next action is planned i.e. argmax.
  \item Planning procedure i.e. evolutional strategy.
  \end{enumerate}
\end{enumerate}

\subsubsection{High-level idea and argument ``why?!''}
They show working example on continuous control tasks of online planning (wheres World Models was offline planning). 
Recurrent state space model: We design a latent dynamics model with both deterministic and stochastic components (Buesing et al., 2018; Chung et al., 2015). Our experiments indicate having both components to be crucial for high planning performance.
Latent overshooting: We generalize the standard variational bound to include multi-step predictions. Using only terms in latent space results in a fast regularizer that can improve long-term predictions and is compati- ble with any latent sequence model.

\subsubsection{Data collection i.e. an iterative approach}
Since the agent may not initially visit all parts of the environment, we need to iteratively collect new experience and refine the dynamics model. We do so by planning with the partially trained model, as shown in Algorithm 1. Starting from a small amount of S seed episodes collected under random actions, we train the model and add one additional episode to the data set every C update steps. When collecting episodes for the data set, we add small Gaussian exploration noise to the action. To reduce the planning horizon and provide a clearer learning signal to the model, we repeat each action R times, as common in reinforcement learning (Mnih et al., 2015; 2016).

\subsubsection{Preprocessing}
\begin{enumerate}
\item action repeat 4
\item action one-hot
\item minimum duration 50
\item maximum duration 2000/4
\item resize (64 x 64 x 3)
\item cast to uint8
\item 5 bit quantisation + random noise data augmentation
\item Cast to from -0.5 to 0.5.
  
\end{enumerate}

\subsubsection{RSSM architecture}
It uses code from official implementation.

\editnote{TODO: Recreate prob. model diagram from your WM vs. PlaNet comparison notes in Draw.io.}
\editnote{TODO: Describe this diagram in probabilistic notation (see note).}

[...] we name recurrent state-space model (RSSM), where f(ht−1, st−1, at−1), deterministic state model, is implemented as a recurrent neural network (RNN). Intuitively, we can understand this model as splitting the state into a stochastic part st and a deterministic part ht, which depend on the stochastic and deterministic parts at the previous time step through the RNN.We use the encoder q(s1:T | o1:T, a1:T) =
?T t=1 q(st | ht, ot) to parameterize the approximate state posteriors. Importantly, all information about the observations must pass through the sampling step of the encoder to avoid a deterministic shortcut from inputs to reconstructions.

\subsubsection{CEM planner}
\editnote{TODO: Cut it down and focus on you modification with cast to one-hot max action.}
We use the cross entropy method (CEM; Rubinstein, 1997; Chua et al., 2018) to search for the best action sequence under the model, as outlined in Algorithm 2. We decided on this algorithm because of its robustness and because it solved all considered tasks when given the true dynamics for planning. CEM is a population- based optimization algorithm that infers a distribution over action sequences that maximize the objective. As detailed in Algorithm 2 in the appendix, we initialize a time-dependent diagonal Gaussian belief over optimal action sequences at:t+H ∼ Normal(µt:t+H, σ2
t:t+HI), where t is the current
time step of the agent and H is the length of the planning horizon. Starting from zero mean and unit variance, we repeatedly sample J candidate action sequences, evaluate them under the model, and re-fit the belief to the top K action sequences. After I iterations, the planner returns the mean of the belief for the current time step, µt. Importantly, after receiving the next observation, the belief over action sequences starts from zero mean and unit variance again to avoid local optima.
To evaluate a candidate action sequence under the learned model, we sample a state trajectory starting from the current state belief, and sum the mean rewards predicted along the sequence. Since we use a population-based optimizer,
we found it sufficient to consider a single trajectory per action sequence and thus focus the computational budget on evaluating a larger number of different sequences. Because the reward is modeled as a function of the latent state, the planner can operate purely in latent space without generating images, which allows for fast evaluation of large batches of action sequences. The next section introduces the latent dynamics model that the planner uses.

\editnote{TODO: Add pseudo-code of planning procedure.}
